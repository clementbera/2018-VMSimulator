%%%% For double-blind review submission, w/o CCS and ACM Reference (max submission space)
\documentclass[10pt, sigplan]{acmart}
%%\settopmatter{printfolios=true,printccs=false,printacmref=false}
%% For double-blind review submission, w/ CCS and ACM Reference
%\documentclass[sigplan,10pt,review,anonymous]{acmart}\settopmatter{printfolios=true}
%% For single-blind review submission, w/o CCS and ACM Reference (max submission space)
%\documentclass[sigplan,10pt,review]{acmart}\settopmatter{printfolios=true,printccs=false,printacmref=false}
%% For single-blind review submission, w/ CCS and ACM Reference
%\documentclass[sigplan,10pt,review]{acmart}\settopmatter{printfolios=true}
%% For final camera-ready submission, w/ required CCS and ACM Reference
%\documentclass[sigplan,10pt]{acmart}\settopmatter{}


%% Conference information
%% Supplied to authors by publisher for camera-ready submission;
%% use defaults for review submission.
%\acmConference[PL'17]{ACM SIGPLAN Conference on Programming Languages}{January 01--03, 2017}{New York, NY, USA}
%\acmYear{2017}
%\acmISBN{} % \acmISBN{978-x-xxxx-xxxx-x/YY/MM}
%\acmDOI{} % \acmDOI{10.1145/nnnnnnn.nnnnnnn}
%\startPage{1}

%% Copyright information
%% Supplied to authors (based on authors' rights management selection;
%% see authors.acm.org) by publisher for camera-ready submission;
%% use 'none' for review submission.
\setcopyright{none}
%\setcopyright{acmcopyright}
%\setcopyright{acmlicensed}
%\setcopyright{rightsretained}
%\copyrightyear{2017}           %% If different from \acmYear

%% Bibliography style

%\bibliographystyle{ACM-Reference-Format}

\citestyle{acmnumeric}   


%%%%%%%%%%%%%%%%%%%%%%%%%%%%%%%%%%%%%%%%%%%%%%%%%%%%%%%%%%%%%%%%%%%%%%
%% Note: Authors migrating a paper from traditional SIGPLAN
%% proceedings format to PACMPL format must update the
%% '\documentclass' and topmatter commands above; see
%% 'acmart-pacmpl-template.tex'.
%%%%%%%%%%%%%%%%%%%%%%%%%%%%%%%%%%%%%%%%%%%%%%%%%%%%%%%%%%%%%%%%%%%%%%


%% Some recommended packages.
\usepackage{booktabs}   %% For formal tables:
                        %% http://ctan.org/pkg/booktabs
\usepackage{subcaption} %% For complex figures with subfigures/subcaptions
                        %% http://ctan.org/pkg/subcaption
\usepackage{xspace}
\usepackage{graphicx}
\usepackage{ifthen}
\usepackage{pgfplots}
\usepackage{listings}
\usepackage{multirow}
\usepgfplotslibrary{statistics}
\usepackage{dblfloatfix} %enable fig at bottom of page
%



% Source Code
%\usepackage{color}
%\usepackage{textcomp}
%\usepackage{listings}
%\usepackage{ulem}
%\usepackage[T1]{fontenc}
%\usepackage{times}
% \usepackage{needspace}
 

% Source Code
\usepackage{color}
\usepackage{textcomp}
\usepackage{listings}

\definecolor{source}{gray}{0.85}% my comment style
\newcommand{\myCommentStyle}[1]{{\footnotesize\sffamily\color{gray!100!white} #1}}
%\newcommand{\myCommentStyle}[1]{{\footnotesize\sffamily\color{black!100!white} #1}}

% my string style
\newcommand{\myStringStyle}[1]{{\footnotesize\sffamily\color{violet!100!black} #1}}
%\newcommand{\myStringStyle}[1]{{\footnotesize\sffamily\color{black!100!black} #1}}

% my symbol style
\newcommand{\mySymbolStyle}[1]{{\footnotesize\sffamily\color{violet!100!black} #1}}
%\newcommand{\mySymbolStyle}[1]{{\footnotesize\sffamily\color{black!100!black} #1}}

% my keyword style
\newcommand{\myKeywordStyle}[1]{{\footnotesize\sffamily\color{green!70!black} #1}}
%\newcommand{\myKeywordStyle}[1]{{\footnotesize\sffamily\color{black!70!black} #1}}

% my global style
\newcommand{\myGlobalStyle}[1]{{\footnotesize\sffamily\color{blue!100!black} #1}}
%\newcommand{\myGlobalStyle}[1]{{\footnotesize\sffamily\color{black!100!black} #1}}

% my number style
\newcommand{\myNumberStyle}[1]{{\footnotesize\sffamily\color{brown!100!black} #1}}
%\newcommand{\myNumberStyle}[1]{{\footnotesize\sffamily\color{black!100!black} #1}}

\lstset{
language={},
% characters
tabsize=3,
escapechar={!},
keepspaces=true,
breaklines=true,
alsoletter={\#},
literate={\$}{{{\$}}}1,
breakautoindent=true,
columns=fullflexible,
showstringspaces=false,
% background
frame=single,
aboveskip=1em, % automatic space before
framerule=0pt,
basicstyle=\footnotesize\sffamily\color{black},
keywordstyle=\myKeywordStyle,% keyword style
commentstyle=\myCommentStyle,% comment style
frame=single,%
backgroundcolor=\color{source},
% numbering
stepnumber=1,
numbersep=10pt,
numberstyle=\tiny,
numberfirstline=true,
% caption
captionpos=b,
% formatting (html)
moredelim=[is][\bfseries]{<b>}{</b>},
moredelim=[is][\textit]{<i>}{</i>},
moredelim=[is][\underbar]{<u>}{</u>},
moredelim=[is][\color{red}\uwave]{<wave>}{</wave>},
moredelim=[is][\color{red}\sout]{<del>}{</del>},
moredelim=[is][\color{blue}\underbar]{<ins>}{</ins>},
% smalltalk stuff
morecomment=[s][\myCommentStyle]{"}{"},
%    morecomment=[s][\myvs]{|}{|},
morestring=[b][\myStringStyle]',
moredelim=[is][]{<sel>}{</sel>},
moredelim=[is][]{<rcv>}{</rcv>},
moredelim=[is][\itshape]{<symb>}{</symb>},
moredelim=[is][\scshape]{<class>}{</class>},
morekeywords={true,false,nil,self,super,thisContext},
identifierstyle=\idstyle,
}

\makeatletter
\newcommand*\idstyle[1]{%
\expandafter\id@style\the\lst@token{#1}\relax%
}
\def\id@style#1#2\relax{%
\ifnum\pdfstrcmp{#1}{\#}=0%
% this is a symbol
\mySymbolStyle{\the\lst@token}%
\else%
\edef\tempa{\uccode`#1}%
\edef\tempb{`#1}%
\ifnum\tempa=\tempb%
% this is a global
\myGlobalStyle{\the\lst@token}%
\else%
\the\lst@token%
\fi%
\fi%
}
\makeatother


%\newcommand{\ct}{\lstinline[backgroundcolor=\color{white}]}
%\newcommand{\needlines}[1]{\Needspace{#1\baselineskip}}
\newcommand{\lct}{\texttt}

\lstnewenvironment{code}{%
    \lstset{%
    % frame=lines,
    frame=single,
    framerule=0pt,
    mathescape=false
    }%
    \noindent%
    \minipage{\linewidth}%
}{%
    \endminipage%
}%


\lstnewenvironment{codeWithLineNumbers}{%
    \lstset{%
    % frame=lines,
    frame=single,
    framerule=0pt,
    mathescape=false,
    numbers=left
    }%
    \noindent%
    \minipage{\linewidth}%
}{%
    \endminipage%
}%



\newenvironment{codeNonSmalltalk}
{\begin{alltt}\sffamily}
{\end{alltt}\normalsize}



\usepackage{xcolor}
\newcommand{\todo}[1]{\color{orange}\fbox{\bfseries\sffamily\scriptsize TODO:}{\sf\small$\blacktriangleright$\textit{#1}$\blacktriangleleft$}\color{black}}
\newcommand{\sd}[1]{\color{red}\fbox{\bfseries\sffamily\scriptsize Stef:}{\sf\small$\blacktriangleright$\textit{#1}$\blacktriangleleft$}\color{black}}
\newcommand{\sk}[1]{\color{blue}\fbox{\bfseries\sffamily\scriptsize Sophie:}{\sf\small$\blacktriangleright$\textit{#1}$\blacktriangleleft$}\color{black}}
\newcommand{\cba}[1]{\color{purple}\fbox{\bfseries\sffamily\scriptsize Clement:}{\sf\small$\blacktriangleright$\textit{#1}$\blacktriangleleft$}\color{black}}
%\newcommand*{rotatebox{75}}




% Source Code
%\usepackage{color}
%\usepackage{textcomp}
%\usepackage{listings}
%\usepackage{ulem}
%\usepackage[T1]{fontenc}
%\usepackage{times}
% \usepackage{needspace}
 

% Source Code
\usepackage{color}
\usepackage{textcomp}
\usepackage{listings}

\definecolor{source}{gray}{0.85}% my comment style
\newcommand{\myCommentStyle}[1]{{\footnotesize\sffamily\color{gray!100!white} #1}}
%\newcommand{\myCommentStyle}[1]{{\footnotesize\sffamily\color{black!100!white} #1}}

% my string style
\newcommand{\myStringStyle}[1]{{\footnotesize\sffamily\color{violet!100!black} #1}}
%\newcommand{\myStringStyle}[1]{{\footnotesize\sffamily\color{black!100!black} #1}}

% my symbol style
\newcommand{\mySymbolStyle}[1]{{\footnotesize\sffamily\color{violet!100!black} #1}}
%\newcommand{\mySymbolStyle}[1]{{\footnotesize\sffamily\color{black!100!black} #1}}

% my keyword style
\newcommand{\myKeywordStyle}[1]{{\footnotesize\sffamily\color{green!70!black} #1}}
%\newcommand{\myKeywordStyle}[1]{{\footnotesize\sffamily\color{black!70!black} #1}}

% my global style
\newcommand{\myGlobalStyle}[1]{{\footnotesize\sffamily\color{blue!100!black} #1}}
%\newcommand{\myGlobalStyle}[1]{{\footnotesize\sffamily\color{black!100!black} #1}}

% my number style
\newcommand{\myNumberStyle}[1]{{\footnotesize\sffamily\color{brown!100!black} #1}}
%\newcommand{\myNumberStyle}[1]{{\footnotesize\sffamily\color{black!100!black} #1}}

\lstset{
language={},
% characters
tabsize=3,
escapechar={!},
keepspaces=true,
breaklines=true,
alsoletter={\#},
literate={\$}{{{\$}}}1,
breakautoindent=true,
columns=fullflexible,
showstringspaces=false,
% background
frame=single,
aboveskip=1em, % automatic space before
framerule=0pt,
basicstyle=\footnotesize\sffamily\color{black},
keywordstyle=\myKeywordStyle,% keyword style
commentstyle=\myCommentStyle,% comment style
frame=single,%
backgroundcolor=\color{source},
% numbering
stepnumber=1,
numbersep=10pt,
numberstyle=\tiny,
numberfirstline=true,
% caption
captionpos=b,
% formatting (html)
moredelim=[is][\bfseries]{<b>}{</b>},
moredelim=[is][\textit]{<i>}{</i>},
moredelim=[is][\underbar]{<u>}{</u>},
moredelim=[is][\color{red}\uwave]{<wave>}{</wave>},
moredelim=[is][\color{red}\sout]{<del>}{</del>},
moredelim=[is][\color{blue}\underbar]{<ins>}{</ins>},
% smalltalk stuff
morecomment=[s][\myCommentStyle]{"}{"},
%    morecomment=[s][\myvs]{|}{|},
morestring=[b][\myStringStyle]',
moredelim=[is][]{<sel>}{</sel>},
moredelim=[is][]{<rcv>}{</rcv>},
moredelim=[is][\itshape]{<symb>}{</symb>},
moredelim=[is][\scshape]{<class>}{</class>},
morekeywords={true,false,nil,self,super,thisContext},
identifierstyle=\idstyle,
}

\makeatletter
\newcommand*\idstyle[1]{%
\expandafter\id@style\the\lst@token{#1}\relax%
}
\def\id@style#1#2\relax{%
\ifnum\pdfstrcmp{#1}{\#}=0%
% this is a symbol
\mySymbolStyle{\the\lst@token}%
\else%
\edef\tempa{\uccode`#1}%
\edef\tempb{`#1}%
\ifnum\tempa=\tempb%
% this is a global
\myGlobalStyle{\the\lst@token}%
\else%
\the\lst@token%
\fi%
\fi%
}
\makeatother


%\newcommand{\ct}{\lstinline[backgroundcolor=\color{white}]}
%\newcommand{\needlines}[1]{\Needspace{#1\baselineskip}}
\newcommand{\lct}{\texttt}

\lstnewenvironment{code}{%
    \lstset{%
    % frame=lines,
    frame=single,
    framerule=0pt,
    mathescape=false
    }%
    \noindent%
    \minipage{\linewidth}%
}{%
    \endminipage%
}%


\lstnewenvironment{codeWithLineNumbers}{%
    \lstset{%
    % frame=lines,
    frame=single,
    framerule=0pt,
    mathescape=false,
    numbers=left
    }%
    \noindent%
    \minipage{\linewidth}%
}{%
    \endminipage%
}%



\newenvironment{codeNonSmalltalk}
{\begin{alltt}\sffamily}
{\end{alltt}\normalsize}



\begin{document}

%%%
% Legend definition (so I can change it once for all)
%%%

\def\legBase{Baseline\xspace}
\def\legSmartSyntaxPlug{Syntax\xspace}
\def\legSlangPlug{Plugin\xspace}
\def\legSlang{Slang\xspace}
\def\legRTL{Slang+RTL\xspace}
\def\LLVMMac{LLVM-Mac\xspace}
\def\GCCLinux{GCC-Linux\xspace}
\def\plotHeight{6.8cm}

%%%
% End Legend
%%%

%% Title information
\title[Over Twenty Years of Virtual Machine Development Through Simulation]{Over Twenty Years of Virtual Machine Development and Debugging Through Simulation}

%% Author with single affiliation.
\author{Eliot Miranda}
                                        %% can be repeated if necessary
\affiliation{
  %\position{Position1}
 % \department{VM team}              %% \department is recommended
  \institution{Feenk}            %% \institution is required
 % \streetaddress{Street1 Address1}
  \city{San Francisco}
  %\state{France}
  %\postcode{Post-Code1}
  \country{California}                    %% \country is recommended
}
\email{eliot.miranda@gmail.com}          %% \email is recommended

%% Author with two affiliations and emails.
\author{Cl\'ement B\'era}
\affiliation{
  % \position{}
	\department{Software Languages Lab}              %% \department is recommended
	\institution{Vrije Universiteit Brussel}            %% \institution is required
	\city{Brussel}
  % \state{}
  % \postcode{}
	\country{Belgium}                    %% \country is recommended
}
\email{clement.bera@vub.ac.be}          %% \email is recommended

%% Author with two affiliations and emails.
\author{Elisa Gonzalez Boix}
\affiliation{
  % \position{}
	\department{Software Languages Lab}              %% \department is recommended
	\institution{Vrije Universiteit Brussel}            %% \institution is required
	\city{Brussel}
  % \state{}
  % \postcode{}
	\country{Belgium}                    %% \country is recommended
}
\email{egonzale@vub.ac.be}          %% \email is recommended


%% Abstract
%% Note: \begin{abstract}...\end{abstract} environment must come
%% before \maketitle command
\begin{abstract}
Fork from Dan Ing VM
Dev in Restr St, Comp through C, simulation by interpreting the Slang.
Evolved through the years with processor simulator and various extensions
We explain how we generate the prod VM and how we simulate it for dev and debugging.
We go through 2 experience example, GC and JIT, and show why we believe how valuable it the infra

\end{abstract}

%% 2012 ACM Computing Classification System (CSS) concepts
%% Generate at 'http://dl.acm.org/ccs/ccs.cfm'.
%\begin{CCSXML}
%<ccs2012>
%<concept>
%<concept_id>10011007.10011006.10011008</concept_id>
%<concept_desc>Software and its engineering~General programming languages</concept_desc>
%<concept_significance>500</concept_significance>
%</concept>
%<concept>
%<concept_id>10003456.10003457.10003521.10003525</concept_id>
%<concept_desc>Social and professional topics~History of programming languages</concept_desc>
%<concept_significance>300</concept_significance>
%</concept>
%</ccs2012>
%\end{CCSXML}

%\ccsdesc[500]{Software and its engineering~General programming languages}
%\ccsdesc[300]{Social and professional topics~History of programming languages}
%% End of generated code


%% Keywords
%% comma separated list
\keywords{Just-in-Time compiler, Virtual machine, Managed runtime, Tools}  %% \keywords are mandatory in final camera-ready submission


%% \maketitle
%% Note: \maketitle command must come after title commands, author
%% commands, abstract environment, Computing Classification System
%% environment and commands, and keywords command.
\maketitle


\section{Introduction}
\label{sec:intro}

\section{Virtual Machine Simulation \& Production Infrastructure}
\label{sec:VMSimulation}

\subsection{Generating the Production Virtual Machine}

\subsection{Simulating the Virtual Machine}

\section{Garbage Collection Development}
\label{sec:GCExp}

default behavior, cloneOnSavenge/GC
Lemming debugging

Levels of Assertions: ST only (debugging code in St), Slang/C, Node.

\section{Just-in-Time Compiler Development}
\label{sec:JITExp}
Back-in-time debugging of mahcine state
Conditional stepping.

\section{Virtual Machine Analysis}
\label{sec:Analysis}
Analysis of caches

Analysis of machine code zone

Analysis of the heap

\section{Discussion, Related Work and Conclusion}

\subsection{Discussion}
Simulation perf and its limitations
Out of the simulation, FFI calls and call-backs

\subsection{Related Work}
Maxine inspectors

RPython toolchain, though it seems they don't do it (Balance abstraction and time to compile/simulate)

\subsection*{Conclusion}


%% Acknowledgments
%%\begin{acks}                            %% acks environment is optional
                                        %% contents suppressed with 'anonymous'
  %% Commands \grantsponsor{<sponsorID>}{<name>}{<url>} and
  %% \grantnum[<url>]{<sponsorID>}{<number>} should be used to
  %% acknowledge financial support and will be used by metadata
  %% extraction tools.
%  This material is based upon work supported by the
%  \grantsponsor{GS100000001}{National Science
%    Foundation}{http://dx.doi.org/10.13039/100000001} under Grant
%  No.~\grantnum{GS100000001}{nnnnnnn} and Grant
%  No.~\grantnum{GS100000001}{mmmmmmm}.  Any opinions, findings, and
%  conclusions or recommendations expressed in this material are those
%  of the author and do not necessarily reflect the views of the
%  National Science Foundation.
%\end{acks}

%% Bibliography
\bibliographystyle{alpha}
\bibliography{sista,rmod,others}

\end{document}
